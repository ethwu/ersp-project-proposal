\section{Introduction}\label{introduction}

    Data and testbeds for networking research are currently largely in the hands of large corporations~\autocite{GuptaArpit2019AEtD}, a problem which may be solved through the use of data collection and testing on campus networks. This use of campus networks for networking research has been done before; for example, at Princeton~University by~\textcite{KimHyojoon2021Erop}. The University of~California, Santa~Barbara Systems and Networking~Lab~(SNL) has been working on a similar project, using a network of Raspberry~Pi clones distributed around campus to test the performance of the campus network when streaming media.

    \Textcite{10.1145/1355734.1355746}~suggest the use of their existing tool called OpenFlow, which enables easy control of the scope of the experiments on the campus networks. Aside from OpenFlow, the peripherals and tools remained underdeveloped. Researchers do not have a tool that they could grab-and-use, but instead they need to make conversations with school officials, and gain access to the data. What's more, they have to spend time developing their own data collection, marking, and analyzation tools, and therefore, have less time to work on the training and gain better results.

    Other technologies than OpenFlow exist for network research, as well. \Textcite{KimHyojoon2021Erop}~present a proof-of-concept of campus network data collection that is based upon the Protocol Independent Switch Architecture~(PISA) and P4.
    % These tools are built upon Software-Defined Networking~(SDN), which allows switches to be controlled from software.
    P4 is a protocol programming language designed to resolve extensibility issues in OpenFlow by abstracting over it at a software level~\autocite{BosshartPat2014Pppp}. Much experimentation and testing in networking reserach is done through simulated networks using software switches~\autocite{KimHyojoon2021Erop}, which may behave differently from real switches operating on real-world data. \Textcite*{KimHyojoon2021Erop}'s~P4Campus uses the programmability of software switches to selectively capture the data most relevant to their research.
