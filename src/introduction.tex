\section{Introduction}\label{introduction}
    It is constant progress that we want to improve the users' quality of experience when using the Internet, and how researchers and the network service providing industry can provide better and more efficient Internet services to users will always be an unresolved issue. In order to improve the quality of experience of the users, we must first understand what are the contributing factors that influence the users' experience, and later, building from these results, researchers or the industry will be able to improve or manipulate the networks to improve the users' quality of experience.

    Recently, there has been a large shift towards online video conferencing due to COVID-19. While much research has been done on video streaming applications, relatively little has been done for video conferencing applications~(VCAs). In particular, some~VCAs make it difficult to evaluate the quality of experience~(QoE) and the quality of service~(QoS) metrics from the network or even application layer. In particular: Zoom, a popular~VCA controlling 76\,\%~of the video-conferencing market as of June~2021~\autocite{kim2021}; uses its own closed-source transport protocol~\autocite{marczak2020} and only reports application performance at a minute's granularity~\autocite{walia2019}. This makes it difficult to evaluate~QoE and~QoS. Only by drawing relations between QoS and QoE metrics, we would be able to know more about possible ways we can further optimize the network and the experience of the users, when the network resources are limited, by updating the algorithms used by the programmable switches and servers. The current progress is done on the end-host machines, which enable us to access the WebRTC data. However, our ultimate goal is to draw relations from packets captured at the border of UCSB to the Quality of Experience of the students, which means we only have access to encrypted data carried in UDP by that time. 

    \subsection{UDP}\label{introduction:qos:udp}
        UDP, User Datagram Protocol, is a lightweight transport protocol, providing minimal services. UDP is connectionless, so there is no handshaking before the two processes start to com- municate. UDP provides an unreliable data transfer service—that is, when a process sends a message into a UDP socket, UDP provides no guarantee that the message will ever reach the receiving process. Furthermore, messages that do arrive at the receiving process may arrive out of order. UDP does not include a congestion-control mechanism, so the sending side of UDP can pump data into the layer below (the network layer) at any rate it pleases.~\autocite{alma990025667610203776} Since UDP is a minimal transport layer protocol, the things we can get would be the source port, the destination port, and the application-layer-encrypted payloads that it carries.

    \subsection{Quality of Service}\label{introduction:qos}
        Quality of service (QoS) is a metric expressing the performance of the network. One important thing we need to know is packets: During video meetings or video converances through Internet, computers need to send and receive data. The data is packed into chunks, which we call packets, and sent through the links and into the Internet. The protocol used by most~VCAs is the User Datagram Protocol (UDP). However, UDP~packets do not contain any information about the network quality. Instead, a limited amount of network information is recorded by the application-layer protocol,~WebRTC.



        \subsubsection{WebRTC}\label{introduction:qos:webrtc}
            WebRTC is an application-layer protocol used for video conferencing. It is built upon~RTP. While some of the popular~VCAs use~WebRTC, Zoom uses its own closed-source extension~\autocite{marczak2020}, making it difficult to evaluate QoE~and QoS~metrics from the network layer. While the Zoom~API does provide some QoE information, it is at a minute's granularity~\autocite{walia2019}, which is both at the application layer and is not fine enough for our purposes. Here aare the elements that WebRTC collects:

            \begin{description}
                \item[Uplink Bitrate] The \emph{uplink bitrate} or \emph{uplink capacity} is the rate at which the client sends packets to other users or to a relay server. It is usually measured in~Mbps. A low uplink bitrate generally leads to poor audio/video upload quality.

                \item[Downlink Bitrate] The \emph{downlink bitrate} or \emph{downlink capacity} is the rate at which the client downloads from other users or from a relay server. It is also measured in~Mbps. A low downlink bitrate generally leads to low resolution video, poor refresh rates, and other drops in video quality.

                \item[Packet Loss] Network data may be lost in transit, quantified as \emph{packet loss}. This may be because of software factors such as network congestion, or physical factors such as poor wireless signal or broken cabling. Packet loss is measured as a percentage over a given time period.
            \end{description}
        

        % There are three known major factors that contribute to poor video quality: packet loss due to unreliable transmission, delays caused by the network capacity, and jitter, caused by irregular delays~\autocite{ChenYanjiao2015FQtQ}.

    \subsection{Quality of Experience}\label{introduction:qoe}
        Quality of experience is the user's perception of~QoS~\autocite{ChenYanjiao2015FQtQ}. Because QoE~is a subjective metric, it is typically measured using opinion surveys over experiment participants~\autocite{ChenYanjiao2015FQtQ,RodrriguezDemóstenesZ2014Vqai}. Since the QoS~metrics do not necessarily translate one-to-one to their perception of service quality, QoE~has supplanted~QoS as the barometer of user satisfaction~\autocite{DinakiHosseinEbrahimi2021FVQW}.

        While surveying users is an accurate measure of~QoE, it is costly, not real-time, and only viable in a laboratory environment~\autocite{ChenYanjiao2015FQtQ,SongHan2011Qpsq}. Because of this, QoE~must generally be modeled using the QoS~metrics available to network operators. One such approach is that of \textcite{DinakiHosseinEbrahimi2021FVQW}, who use a heuristic for video streaming~QoE factoring the playback duration, join time buffering, buffering length, buffering frequency, and average bitrate. \Textcite{SongHan2011Qpsq} describe another QoE~metric, Q-score. These systems use the QoS~data in order to infer the user's~QoE in real-time, allowing the network operator to make decisions on how to optimize the network performance.

        Video conferencing in particular is impacted by QoS~metrics such as the \emph{frames per second}~(FPS), \emph{quantization parameter}, and the \emph{video resolution}~\autocite{MacMillanKyle2021MtPa}.
        % video conferencing uses different qoe metrics, find a paper talking about those

        % If this paragraph is eventually relevant to our solution, keep it.
        Quality of experience can be impaired by visual distortions. In video streaming, these distortions may include~\autocite{YuenMichael1998Asoh}:
        \begin{description}
            \item[Blocking Effect] The reconstructed image contains block-shaped discontinuities due to the frame being encoded in blocks.
            \item[Blurring] The amount of detail in the image is lost, and the sharpness of edges is reduced.
            \item[Color Bleeding] Detail is smoothed out and areas with contrasting colors show smears of the other color.
            \item[Edginess] The edges of the image show distortions~\autocite{ChenYanjiao2015FQtQ}.
            \item[Jerkiness] \enquote{Snapshots} of a continuous motion appear in succession, producing a \enquote{disjointed sequence}.
        \end{description}

        Quality of experience can also be negatively impacted by the format with which the video is compressed. When automatically monitoring for~QoE, systems often consider the bitrate and the frame rate~\autocite{ChenYanjiao2015FQtQ}. The bitrate is the rate at which the video codec compressing the video emits data, and the frame rate is the number of frames displayed per second. Moreover, the video itself may impact the~QoE---videos with low amounts of movement may appear to have higher quality than videos with high amounts of movement when subjected to jitter or packet loss, for example~\autocite{ChenYanjiao2015FQtQ}.


    % This section covers our motivations for the project.
    %\subsection{Applications}\label{introduction:applications}
    \subsection{Proactive QoE Management}\label{introduction:applications:management}
        Real-time QoE~monitoring can be used to improve the perceived~QoE of users on the network. For example, the network may use the QoS~parameters to proactively estimate the user's~QoE, and choose to reprioritize currently buffered packets in order to optimize said~QoE~\autocite{DinakiHosseinEbrahimi2021FVQW}.


    We intend to use data collected passively from UCSB's~network in order to create machine learning models for users' quality of experience using video conferencing from packet capture data available to the network operator.
