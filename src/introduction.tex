\section{Introduction}\label{introduction}

    % It is constant progress that we want to improve the users' quality of experience when using the Internet, and how researchers and the network service providing industry can provide better and more efficient Internet services to users will always be an unresolved issue. In order to improve the quality of experience of the users, we must first understand what are the contributing factors that influence the users' experience, and later, building from these results, researchers or the industry will be able to improve or manipulate the networks to improve the users' quality of experience.

    Recently, there has been a large shift towards online video conferencing for school and work, because of COVID-19. An unprecedented amount of network usage is now being committed to video conferencing. Because so many activites are now being conducted through video conferencing applications~(VCAs), it is more important than ever to understand and improve the video conferencing experience.

    The network operator is in a unique position to improve video conferencing for users of the network, without requiring any direct involvement from users or VCA~vendors such as Zoom (whose namesake~VCA controls 76\,\%~of the video-conferencing market as of June~2021~\autocite{kim2021}), Google (Google~Meet), or Microsoft (Microsoft~Teams). With a machine learning model, the network operator can use network quality of service~(QoS) statistics, such as the uplink and downlink bitrates, to infer VCA~users' quality of experience~(QoE)---that is, are they experiencing stuttering, poor framerate, blurry video, etc. This has previously been done for other network-intensive services, such as this work on video streaming by~\textcite{ChenYanjiao2015FQtQ}. Real-time QoE~inference and monitoring can be used to improve video conferences through tactics available to the network operator, such as reprioritizing currently buffered packets to favor users estimated to be suffering from poor performance~\autocite{DinakiHosseinEbrahimi2021FVQW}.

    We intend to leverage UCSB's~campus network in order to collect data that will allow us to create machine learning models capable of predicting users' quality of experience from the packet capture data available to the network operator. With these models, network operators will be able to write improved network routing and queuing algorithms to optimize users' video conferencing experiences. % We believe that this research will be helpful for future developments of regional network control and optimization algorithms since it provides the important mapping from raw packet capture data to users' Quality of experience for Video Conferancing Applications.

    \subsection{VCA Protocols}\label{introduction:qos:udp}
        Modern VCAs use the User Datagram Protocol~(UDP) together with an application-layer protocol in order to transmit data for the conference. Zoom, the most popular~VCA~\autocite{kim2021}, uses a proprietary application-layer protocol~\autocite{marczak2020}, while Google~Meet and Microsoft~Teams use the open standard~WebRTC.

        UDP is a lightweight transport protocol that foregoes reliability for speed. The Transmission Control Protocol~(TCP) is the other major transport protocol. It prioritizes having every packet of data arrive at its destination. UDP,~on the other hand, does not provide any guarantee that a message will reach the receiving process. Furthermore, while TCP~ensures that packets arrive in the correct order, under~UDP packets may arrive in any order. Finally, UDP~does not include any congestion control, meaning that there is no limit on the number of messages being sent~\autocite{alma990025667610203776}. This may overwhelm the network if too many packets are sent.

        Because the speed of~UDP is desirable, VCAs~use application-layer protocols built on top of~UDP in order to send conferencing data such as audio, and video. Google~Meet and Microsoft~Teams use WebRTC, a project containing a set of these real-time communication protocols. % built upon~RTP.

    \subsection{Quality of Service}\label{introduction:qos}
        The performance of the network is quantified in quality of service~(QoS) metrics. Application-layer protocols such as those provided by~WebRTC may report these statistics. WebRTC reports the following QoS~statistics:

        \begin{description}
            \item[Uplink Bitrate/Uplink Capacity] The rate at which the client sends packets to other users or to a relay server, measured in~Mbps. A low uplink bitrate generally leads to poor audio/video upload quality.

            \item[Downlink Bitrate/Downlink Capacity (Mbps)] The rate at which the client downloads from other users or from a relay server, measured in~Mbps. A low downlink bitrate generally leads to low resolution video, poor refresh rates, and other drops in video quality.

            \item[Packet Loss] Network data may be lost in transit, quantified as \emph{packet loss} and recorded as percent lost over a given. This may be because of software factors such as network congestion, or physical factors such as poor wireless signal or broken cabling.
        \end{description}

        Google~Meet and Microsoft~Teams report these data on the user's device. The Zoom~API also reports QoS~and QoE~(see below)~data. However, Zoom only provides this data in minute-long \enquote{chunks}~\autocite{walia2019}, which is too long to be useful in real time. If it takes a minute for the system to learn of a QoE~disruption, the network conditions that led to the problem might already be long gone, because of how quickly the network operates (i.e., much faster than one minute). For example, if the frame rate drops for ten seconds, that disruption is not significant enough to appear in the minute-long report. Important information is thus buried because of the one-minute granularity. Finally, these statistics are only known at the application layer, as the video conferencing data is encrypted when sent through~UDP. The network operator can only see the origin and destination of each packet.

        % One important thing we need to know is packets: During video meetings or video conferances through Internet, computers need to send and receive data. The data is packed into chunks, which we call packets, and sent through the links and into the Internet. The protocol used by most~VCAs is the User Datagram Protocol (UDP). However, UDP~packet headers do not contain any information about the network quality. Instead, a limited amount of network information is recorded by the application-layer protocol,~WebRTC.

        % While much research has been done on video streaming applications, relatively little has been done for video conferencing applications~(VCAs). In particular, some~VCAs make it difficult to evaluate the quality of experience~(QoE) and even the quality of service~(QoS) metrics from the network or even application layer. In particular: Zoom, a popular~VCA controlling 76\,\%~of the video-conferencing market as of June~2021~\autocite{kim2021}; uses its own closed-source transport protocol~\autocite{marczak2020} and only reports application performance at a minute's granularity~\autocite{walia2019}. This makes it difficult to evaluate~QoE and~QoS. Only by drawing relations between raw packet capture data and QoE metrics, we would be able to know more about possible ways we can further optimize the network and the experience of the users, when the network resources are limited, by updating the algorithms used by the programmable switches and servers. The current progress is done on the end-host machines, which enable us to access the WebRTC data, and these QoS data will help us analyze the contributing factors that influence the QoE. However, our ultimate goal is to draw relations from packets captured at the border of UCSB to the Quality of Experience of the students, which means we only have access to encrypted data carried in UDP by that time.

        % There are three known major factors that contribute to poor video quality: packet loss due to unreliable transmission, delays caused by the network capacity, and jitter, caused by irregular delays~\autocite{ChenYanjiao2015FQtQ}.

    \subsection{Quality of Experience}\label{introduction:qoe}
        Quality of experience is the user's perception of~QoS~\autocite{ChenYanjiao2015FQtQ}. Because of how subjective this perception may be, QoE~is typically measured using opinion surveys over experiment participants~\autocite{ChenYanjiao2015FQtQ,RodrriguezDemóstenesZ2014Vqai}. Since the QoS~metrics do not necessarily translate one-to-one to their perception of service quality, QoE~has supplanted~QoS as the canonical indication of user satisfaction~\autocite{DinakiHosseinEbrahimi2021FVQW}.

        While surveying users is an accurate measure of~QoE, it is costly, not in real-time, and impossible outside of a laboratory environment~\autocite{ChenYanjiao2015FQtQ,SongHan2011Qpsq}. Because of this, QoE~must generally be modeled using the QoS~metrics available to network operators. This has previously been done for video streaming; for example, by \textcite{DinakiHosseinEbrahimi2021FVQW}, who use a heuristic for video streaming~QoE factoring the playback duration, join time buffering, buffering length, buffering frequency, and average bitrate. \Textcite{SongHan2011Qpsq} describe another QoE~metric, Q-score. These systems use the QoS~data in order to infer the user's~QoE in real-time, allowing the network operator to make decisions on how to optimize the network performance. While these examples for video streaming exist, no such research has been done on how to map network statistics for video conferencing.

        % video conferencing uses different qoe metrics, find a paper talking about those
        Video conferencing is impacted by hard statistics such as the \emph{frames per second}~(FPS), \emph{quantization parameter}, and the \emph{video resolution}~\autocite{MacMillanKyle2021MtPa}. Issues may manifest to the user in the form of the following or other distortions~\autocite{YuenMichael1998Asoh}:

        \begin{description}
            \item[Blocking Effect] The reconstructed image contains block-shaped discontinuities due to the frame being encoded in blocks.
            \item[Blurring] The amount of detail in the image is lost, and the sharpness of edges is reduced.
            \item[Color Bleeding] Detail is smoothed out and areas with contrasting colors show smears of the other color.
            \item[Edginess] The edges of the image show distortions~\autocite{ChenYanjiao2015FQtQ}.
            \item[Jerkiness] \enquote{Snapshots} of a continuous motion appear in succession, producing a \enquote{disjointed sequence}.
        \end{description}

        Quality of experience can also be negatively impacted by the format with which the video is compressed. When automatically monitoring for~QoE, systems often consider the bitrate and the frame rate~\autocite{ChenYanjiao2015FQtQ}. The bitrate is the rate at which the video codec compressing the video emits data, and the frame rate is the number of frames displayed per second. Moreover, the video itself may impact the~QoE---videos with low amounts of movement may appear to have higher quality than videos with high amounts of movement when subjected to jitter or packet loss, for example~\autocite{ChenYanjiao2015FQtQ}.
