\section{Introduction}\label{introduction}
    Data and testbeds for networking research are currently largely in the hands of large corporations~\autocite{GuptaArpit2019AEtD}, a problem which may be solved through the use of data collection and testing on campus networks. This use of campus networks for networking research has been done before; for example, at Princeton~University by~\textcite{KimHyojoon2021Erop}. The University of~California, Santa~Barbara Systems and Networking~Lab~(SNL) has been working on a similar project, using a network of Raspberry~Pi clones distributed around campus to test the performance of the campus network when streaming media. Also, Standford University suggest the use of their existing tool called OpenFlow, which enables easy control of the scope of the experiments on the campus networks.~\autocite{10.1145/1355734.1355746} Aside from OpenFlow, the peripherals and tools remained underdeveloped. Researchers do not have a tool that they could grab-and-use, but instead they need to make conversations with school officials, and gain access to the data. What's more, they have to spend time developing their own data collection, marking, and analyzation tools, and therefore, have less time to work on the training and gain better results.
