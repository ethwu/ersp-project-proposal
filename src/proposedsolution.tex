\section{Proposed Solution}\label{proposed-solution}
    In order to predict VCA~QoE from network~QoS, we will first actively collect training data from VCAs using a setup similar to that used by~\textcite{MacMillanKyle2021MtPa}. We will then use statistical analysis to determine which QoS~parameters have the greatest impact on users'~QoE. Finally, we will create a machine learning model using this data to predict~QoE when only network data is available.

    In order to collect data, we will use the QoE~and QoS~data reported by the VCAs themselves. Google~Meet and Microsoft~Teams, for example, provide WebRTC~data dumps including statistics such as frame rate, change in frame rate, and the resolution. We will also collect QoS~data from network packet captures, providing us with packet loss, available bandwidth, and upstream/downstream data rates. With these data, we will find their distributions and determine which QoS~parameters have the greatest impact on~QoE.

    We will then train a machine learning model over these data, taking network metrics as an input and outputting the predicted user~QoE. We expect it to be difficult to find a correct and well-fitted mapping, as the data from the application may not directly reflect the realities of the network.

    % Later, in order to predict QoE from raw packet capture data, we will perform machine learning training, categorizing sessions of each end-host into different categories of QoS. For example, if an end-host experiences packet loss due to a bad network, the packets containing WebRTC data will still contain that information, though it is encrypted. Machine learning will implicitly hold on to that information and put it into the category of high packet loss. In such a way, we create a general mapping from raw data to QoS metrics. By merging our result from the previous step, we are able to monitor the QoE base on packets captured at the network border and provide a path for other researchers, or our professor's other research group, to implement network optimizing algorithms on top of it.

    % i'm not sure what this means, so i wrot ethat the application's data may
    % not accurately reflect the network conditions

    % We expect more difficulties in the second part of our research since it would be hard to find the correct Machine Learning model and obtain a correct and well-fitted mapping from raw data to QoS. The relationships will be tough to draw because the packets also contain information about the videos and audios of the meeting, which are redundancies for our analysis. The encryptions from the application layer will mix the real payloads with the WebRTC data, and create some obstacles for our analysis.
