\section{Proposed Solution}\label{proposed-solution}
    In order to map quality of service to quality of experience for video conferencing applications, we intend to collect QoS~data from the network and use machine learning models to correlate it to QoE~data obtained experimentally from devices.

    The QoS~data will be collected and mark with time stamps; the QoE, which is trickier, will be collected from the local machines. On the one hand, QoS data could be collected mainly through~WebRTC. WebRTC~provides us with very important data such as available bandwidth, resolution, and upstream/downstream datarates. By keeping track of WebRTC~statistics and marking them with timestamp, we can later put the information together and see the relationship between these WebRTC data and the QoE data.

    On the other hand, QoE~is difficult to evaluate. Most VCAs do not provide QoE~data to users and researchers. We will design a tool to capture useful QoE data from the client interface, marking it with timestamps for later processing. These data will include frame rate and change in frame rate.
