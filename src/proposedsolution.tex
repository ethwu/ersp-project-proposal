\section{Proposed Solution}\label{proposed-solution}
    In order to map encrypted packet capture data to quality of experience for video conferencing applications, we separate our research into two major parts: Firstly, collect QoS~data from the network and perform statistical analysis or even machine learning models to correlate it to QoE~data obtained experimentally from devices. Secondly, use the ML model to categorize the encrypted data into different categories of QoS status. By combining these two sub-goals, we can draw a mapping from encrypted packet capture data to the Quality of experience of the users.

    To control and optimize QoE using QoS, we will first actively collect QoE~and QoS~data from video conferencing applications such as Google~Meet and Microsoft~Teams. The QoE~data will comprise statistics reported by the VCAs' respective WebRTC data dumps, including such statistics as frame rate, change in frame rate, and resolution. The QoS~data will comprise packet capture data from the network, including such statistics as packet loss, available bandwidth, and upstream/downstream data rate. We will then characterize these data by drawing visualizations of their probability and cumulative density functions, in order to find their distributions. With this information, we will be able to determine which QoS~parameters impact~QoE. Once the parameters that impact~QoE are known, we will be able to set up a machine learning model to predict~QoE when passed this parameter but omitting factors that may be unavailable to the network operator.

    Later, in order to predict QoE from raw packet capture data, we will perform machine learning training, categorizing sessions of each end-host into different categories of QoS. For example, if an end-host experiences packet loss due to a bad network, the packets containing WebRTC data will still contain that information, though it is encrypted. Machine learning will implicitly hold on to that information and put it into the category of high packet loss. In such a way, we create a general mapping from raw data to QoS metrics. By merging our result from the previous step, we are able to monitor the QoE base on packets captured at the network border and provide a path for other researchers, or our professor's other research group, to implement network optimizing algorithms on top of it. 