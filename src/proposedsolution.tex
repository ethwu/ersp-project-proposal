\section{Proposed Solution}\label{proposedsolution}
    To enable the mapping from the Quality of Service (QoS) to the Quality of Experience (QoE) under the topic of Video Converancing Applications (VCAs), we will create a tool and a environment that allow us to train and use the state-of-the-art AI/ML models to draw links between them. The QoS data should be collected and mark with time stamps; the QoE, which is trickier, will be collected from the local machines. 

    On the one hand, QoS data could be collected mainly through WebRTC. WebRTC proves us very important data such as avaialable bandwidth, resolution and upstream/downstream datarates. By keeping track of WebRTC statistics and marking them with timestamp, we can later put the information together and see the relationship between these WebRTC data and the QoE data.

    On the other hand, QoE is difficult to evaluate. Most of the VCAs do not open their data of QoE, such as frame rate, Bit rate, and video-sound sync rate, etc., to the users and researchers. We will designed a tool to capture useful QoE data from the client interface, such as frame rate, change in frame rate, and mark those data with timestamp such that we can match the QoE data with QoS data and process them later.  
