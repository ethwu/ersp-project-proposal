\section{Proposed Solution}\label{proposed-solution}
    In order to map quality of service to quality of experience for video conferencing applications, we intend to collect QoS~data from the network and perform statistical analysis or even machine learning models to correlate it to QoE~data obtained experimentally from devices.

    We will first actively collect QoE~and QoS~data from video conferencing applications such as Google~Meet and Microsoft~Teams. The QoE~data will comprise statistics reported by the VCAs' respective WebRTC data dumps, including such statistics as frame rate, change in frame rate, and resolution. The QoS~data will comprise packet capture data from the network, including such statistics as packet loss, available bandwidth, and upstream/downstream data rate. We will then characterize these data by drawing visualizations of their probability and cumulative density functions, in order to find their distributions. With this information, we will be able to determine which QoS~parameters impact~QoE. Once the parameters that impact~QoE are known, we will be able to set up a machine learning model to predict~QoE when passed this parameter but omitting factors that may be unavailable to the network operator.
