\section{Evaluation Plan}\label{evaluation}
    Our model will be acceptable if we are able to collect enough data to generalize our findings to other networks under similar conditions. One metric to ensure the validity of our data will be the packet capture rate: If there is a drop in the number of packets captured, we will know that there may be an issue with our data collection. We will be able to use metrics such as accuracy, precision, and recall in order to evaluate the quality of our machine learning model. We must also be wary of the possibility that the model may over- or underfit the data, which would prevent it from being generalizable to other data. This can be tested by running the model on a private router and manipulating the~QoS (i.e., network conditions) ourselves and over test sets collected from the real campus network.

    % The earlier goal is to determine how well the trends of QoE~data match with QoS~data. The way we determine this would be general statistical analysis, which analyzes the relationships, relatabilities, and confidence intervals. By generating plots and graphs, we should be able to view the result very clearly. Moreover, if Machine Learning models are implemented, we can always use metrics, like accuracy, precision, and recall, to evaluate the models. Furthermore, we can always capture packets and generate new training and test sets on our test router, for further training and testing.
