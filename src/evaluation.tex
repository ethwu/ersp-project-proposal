\section{Evaluation Plan}\label{evaluation}
    In order to evaluate our approach, we need to determine if we have collected enough data that we can generalize our result to a greater scope. The first goal of our project is to ensure that we have reliable data that are significant enough for our research. By having a metric that keeps track of the packet capture rate, we can immediately tell if there will be problems with the data we collected when the capture rate drops.

    The most important goal is to determine how well the trends of QoE~data match with QoS~data. The way we determine this would be general statistical analysis, which analyze the relationships, relatabilities, and confidence intervals. By generating plots and graphs, we should be able to view the result very clearly.

    We hope that our project is reproducible. Everybody can collect their own QoS~and QoE~data, given our model of collecting those data, and use their own AI/ML~training models to achieve a similar result while customized for their own network environment. For our result, we expect that the correlations generated by AI/ML~training will not vary much over time, and the trained model should pass the test set generated using later collected data.
